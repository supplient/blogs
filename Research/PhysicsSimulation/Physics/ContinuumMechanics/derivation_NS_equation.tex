\documentclass[lang=cn, a4paper,chinesefont=founder,bibend=bibtex]{elegantpaper}
\title{ElegantPaper: 一个优美的 \LaTeX{} 工作论文模板}
\author{Supplient}
\institute{\href{https://supplient.github.io/blogs/}{My Blog}}
\date{2022/11/4}

% 本文档命令
\usepackage{array}
\newcommand{\ccr}[1]{\makecell{{\color{#1}\rule{1cm}{1cm}}}}
\addbibresource[location=local]{refs.bib} % 参考文献,不要删除
\usepackage{bm}


\begin{document}

\maketitle

\begin{abstract}
本文完全基于\cite{gonzalezFirstCourseContinuum2008}进行Navier-Stokes公式的推导。
\keywords{Navier-Stokes Equations, Continuum Mechanics}
\end{abstract}

\section{Balance Laws}
如果只考虑等温流体,则可以忽略热力学相关的温度(temperature)、热(heat),
则一共有$3+1+9=13$个未知数需要求解:

$$
\begin{aligned}
& v_i & 速度 & \quad \text{3 unknowns} \\
& \rho & 密度 & \quad \text{1 unknown} \\
& S_{ij} & 应力 & \quad \text{9 unknowns} \\
\end{aligned}
$$

而为了求解这13个未知数,我们手头有的公式目前只有balance law。
又因为忽略了热力学相关因素,所以Balance Laws里的热力学第一定律被丢掉了。
于是Balance Laws中剩下的公理都是和运动学相关的了(See Section 5.3): 

$$
\left\{
\begin{aligned}
%
& \dot{\rho} + \rho\nabla^x\cdot \bm{v} = 0 & [\text{质量守恒}] \\
& \rho\dot{\bm{v}} = \nabla^x\cdot \bm{S} + \rho \bm{b} & [\text{线动量守恒}] \\
& S^T = S & [\text{角动量守恒}] \\
%
\end{aligned}
\right.
$$

这里我们选取的是Eulerian Form of Balance Laws,似乎在流体仿真的时候比较流行用Eulerian Form。

上述三个公式提供了$1+3+3=7$个等式,所以为了求解13个未知数,还需要$13-7=6$个等式。
这6个等式将由本构方程(constitutive equation)提供。

\section{Constitutive Model}
称一个连续体为不可压缩的牛顿流体,若:
\begin{enumerate}
	\item 密度均匀:$\rho_0(X, t)=\rho_0>0(\text{constant})$
	\item 不可压缩:$\nabla^x\cdot \bm{v} = 0$
	\item 应力满足:$\bm{S}=-p\bm{I}+2\mu \mathrm{sym}(\nabla^x\bm{v})$
\end{enumerate}

上式三个条件即为不可压缩牛顿流体的本构方程。

\section{Constraints for Constitutive Model}
\begin{itemize}
	\item Result 6.8中验证了在该本构方程下连续体的Frame-Indifference性质依然被满足。
	\item Section 6.3.4中验证了在该本构方程下热力学第二定律依然被满足。
\end{itemize}



\section{简化公式}
联立Balance Laws和Constitutive Model:

$$
\left\{
\begin{aligned}
%
& \dot{\rho} + \rho\nabla^x\cdot \bm{v} = 0 & (\text{质量守恒}) \\
& \rho\dot{\bm{v}} = \nabla^x\cdot \bm{S} + \rho \bm{b} & (\text{线动量守恒}) \\
& S^T = S & (\text{角动量守恒}) \\
%
& \rho_0(X, t)=\rho_0>0(\text{constant}) & (\text{密度均匀}) \\
& \nabla^x\cdot \bm{v} = 0 & (\text{不可压缩}) \\
& \bm{S}=-p\bm{I}+2\mu \mathrm{sym}(\nabla^x\bm{v}) & (\text{应力条件}) \\
%
\end{aligned}
\right.
$$

其中$\bm{b}(x, t), \rho_0, \mu$为给定的body force field per unit mass, density, absolute viscosity。
注意$\rho_0, \mu$都为与$x, t$无关的常数。

\subsection{角动量守恒}
注意到对式(应力条件)两边取转置会得到:

$$
\bm{S}^T = -p\bm{I} + 2\mu\mathrm{sym}(\nabla^x v)
$$

所以式(应力条件)隐含了$\bm{S}=\bm{S}^T$,已经隐含了式(角动量守恒)。


\subsection{质量守恒}
将式(不可压缩)$\nabla^x\cdot \bm{v}$代入式(质量守恒)中会得到:

\[
\dot{\rho} = \frac{\mathrm{d}\rho}{\mathrm{d}t} = 0
\]

即$\rho$不随$t$发生变化,
所以$\rho(x, t) = \rho_m(X, 0)|_{X=\psi(x, t)}$,
代入式(密度均匀),即有:

$$
\begin{aligned}
\rho(x, t) = \rho_0 > 0 (\text{constant}) & \text{(常数密度)}
\end{aligned}
$$

上式隐含了式(质量守恒)和式(密度均匀)。

\subsection{线动量守恒}
将式(应力条件)和式(常数密度)代入式(线动量守恒)中,消掉$\bm{S}, \rho$,得到:

\begin{equation}
	\begin{aligned}
		\rho_0\dot{\bm{v}} 
		&= \nabla^x\cdot (-p\bm{I}+2\mu \mathrm{sym}(\nabla^x\bm{v})) + \rho_0 \bm{b} \\
		&= -\nabla^x\cdot (p\bm{I}) + 2\mu \nabla^x \cdot \mathrm{sym}(\nabla^x\bm{v}) + \rho_0 \bm{b} \\
	\end{aligned}
	\tag{1}
\end{equation}

其中各项有:

$$
\left\{
\begin{aligned}
& \dot{\bm{v}} = \frac{\partial}{\partial t} \bm{v} + (\nabla^x \bm{v}) \bm{v} & \text{[Result 4.7]} \\
& \nabla^x\cdot(p\bm{I}) = p\nabla^x\cdot\bm{I} + \bm{I}\nabla^xp = \nabla^xp & \\
& \nabla^x\cdot\mathrm{sym}(\nabla^x\bm{v}) = \frac{1}{2} ( \nabla^x \cdot \nabla^x\bm{v} + \nabla^x \cdot (\nabla^x\bm{v})^T) = \frac{1}{2} \Delta^x\bm{v} & \text{[See Below]}\\
\end{aligned}
\right.
$$

展开其中对$\nabla^x\cdot\mathrm{sym}(\nabla^x\bm{v})$的推导:

$$
\begin{aligned}
& \nabla^x \cdot \nabla^x \bm{v} = \Delta^x \bm{v} & \text{[拉普拉斯算子的定义]} \\
\end{aligned}
$$

$$
\begin{aligned}
\nabla^x \cdot (\nabla^x\bm{v})^T 
&= \frac{\partial(\nabla^x \bm{v})^T_{ij}}{\partial x_j}\bm{e}_i & \text{[散度定义]} \\
&= \frac{\partial(\frac{\partial v_j}{\partial x_i})}{\partial x_j}\bm{e}_i  & \text{[梯度定义]} \\
&= \frac{\partial^2 v_j}{\partial x_i\partial x_j}\bm{e}_i  & \\
&= \frac{\partial(\frac{\partial v_j}{\partial x_j})}{\partial x_i}\bm{e}_i  & \text{[任意阶导数连续]} \\
&= \nabla^x(\frac{\partial v_j}{\partial x_j})  & \text{[梯度定义]} \\
&= \nabla^x(\nabla^x \cdot \bm{v})  & \text{[散度定义]} \\
&= \nabla^x\bm{0} & \text{[不可压缩]} \\
&= 0 & \\
\end{aligned}
$$

将这三项带回式1中,得到式(线动量守恒)化简后的结果:

\begin{equation}
	\begin{aligned}
		\rho_0 \left[\frac{\partial}{\partial t} \bm{v} + (\nabla^x \bm{v}) \bm{v} \right]
		&= -\nabla^xp + \mu \Delta^x\bm{v} + \rho_0 \bm{b} \\
	\end{aligned}
	\tag{2}
\end{equation}

上式隐含了式(应力条件)、式(常数密度)、式(线动量守恒)。


\subsection{整理公式}
最开始通过联立Balance Laws与Constitutive Model得到的方程组经过前文的化简后变为:
(联立式(不可压缩)和式2)

$$
\left\{
\begin{aligned}
\rho_0 \left[\frac{\partial}{\partial t} \bm{v} + (\nabla^x \bm{v}) \bm{v} \right]
&= -\nabla^xp + \mu \Delta^x\bm{v} + \rho_0 \bm{b} \\
\nabla^x\cdot \bm{v} &= 0 \\
\end{aligned}
\right.
$$

上式即为\textbf{Navier-Stokes Equations}。

其中:

\begin{itemize}
	\item $\bm{b}, \rho_0, \mu$为给定的body force field per unit mass, density, absolute viscosity。 
	\item 一共有4个等式和4个未知数:$p, \bm{v}$,因此可以解出来(笑)。
\end{itemize}















\nocite{*}
\printbibliography[heading=bibintoc, title=\ebibname]

\appendix
%\appendixpage
\addappheadtotoc

\end{document}